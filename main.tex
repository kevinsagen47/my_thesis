%-----------------------------------------------------------------------------------------------------
% MAIN PROGRAM OF THESIS
%-----------------------------------------------------------------------------------------------------

% Set the class of document for NYCU-Thesis
% Availbale Class:
%   模式:[draft] | final (初稿 | 定稿)
%   用途:[print] | upload (輸出 | 上傳)
\documentclass[draft, upload]{Class/NYCU-Thesis}

%-----------------------------------------------------------------------------------------------------
% 參數設定們
%-----------------------------------------------------------------------------------------------------

% 請去Config/config.tex填一些關於這本論文的參數
%----------------------------------------------------------------------
% CONFIGURATION
% 這邊就是一些等一下模板在生出像是封面這些東西的時候,會需要用到的參數
%----------------------------------------------------------------------

% 中英文論文題目
\zhTitle{中文論文題目}
\enTitle{English Thesis Title}

% 中英文關鍵字
%       依據學校規定,關鍵詞 5-7 個,應附於摘要內
\zhKeywords{關鍵字一、關鍵字二、關鍵字三、關鍵字四、關鍵字五}
\enKeywords{Keyword1, Keyword2, Keyword3, Keyword4, Keyword5}

% 研究生中英文姓名
%       依據學校提供的範例,英文姓名應寫「姓, 名」,例:Wang, Jing
\zhStudentName{中文研究生姓名}
\enStudentName{English Name}

% 指導教授中英文姓名
%       依據學校提供的範例,英文姓名應寫「姓, 名」,例:Wang, Jing
\zhAdvisorName{指導教授姓名}
\enAdvisorName{Advisor's English Name}

% 中英文學校名稱
\zhUnivName{國立陽明交通大學}
\enUnivName{National Yang Ming Chiao Tung University}

% 中英文學院名稱
\zhCollegeName{資訊學院}
\enCollegeName{College of Computer Science}

% 中英文研究所名稱
\zhInstName{資訊科學與工程研究所}
\enInstName{Institute of Computer Science and Engineering}

% 英文領域名稱
%       書名頁要用的
\enField{Network Engineering}

% 英文地點名稱
%       書名頁要用的
%       依據學校提供的範例,Taiwan, Republic of China
\enLocation{Taiwan, Republic of China}

% 論文日期
\zhDegreeYear{一一○}
\zhDegreeMonth{七}
\enDegreeYear{2021}
\enDegreeMonth{July}

% 論文浮水印
% 抱歉,窩沒做這個功能,因為窩沒有校徽可以放QQ
% 請去Config/fonts.tex填一下要用的字體
%-------------------------------------------------------------------------------
% FONT SETTINGS
% 拜託填一下要用的中文字型
%-------------------------------------------------------------------------------

% 中文字體
%       其實學校沒規定,內頁的字型要用啥,但是應該大部分都是用「標楷體」才對。
%       請填上系統的標楷體的名稱是是啥
%           Windows:標楷體
%           Overleaf:cwTeXKai
%           其他系統:窩不知道
\zhFont{標楷體}

% 英文字體
%       其實學校沒規定,內頁的字型要用啥,但是應該大部分都是用「Times New Roman」才對。
%       請填上系統的對應的名稱是是啥
\enFont{Times New Roman}

%-----------------------------------------------------------------------------------------------------
% 開始寫內容啦
%-----------------------------------------------------------------------------------------------------

% 這個模板用的Bibliography管理器是biblatex
% biblatex規定要在\begin{document}前加入bib資料
\addbibresource{6-Reference/thesis.bib}

\begin{document}

% 以下註解的數字編號是參考自
% https://aa.nycu.edu.tw/reg/regulation/ 
% 底下的 博碩士學位論文格式規範(中、英文說明)。
% 窩有留一份在Others裡面

% 1. 封面頁
\makeCoverPage

% 2. 書名頁
\makeTitlePage

% 3. 論文電子檔著作權授權書
%       - 這邊提供的是學校的公版文件:
%           https://aa.nycu.edu.tw/reg/regulation/
%       - 口試完將修改完的論文檔案上傳到:
%           https://etd.lib.nctu.edu.tw/cgi-bin/gs32/tugsweb.cgi?o=dwebmge
%           就會拿到填好的各種授權書,所以這是輸出且為定稿才會出現的東西。
%       - 如果有多份授權書需加入,如授權書與延後公開申請書,
%           請先合併成一個PDF然後在這邊指定路徑。
\makeAuthPage{1-Authorization/1-Authorization.pdf}

% 4. 博士論文指導教授推薦書(碩士論文免附)
%       - 窩只有碩士畢業,所以窩沒有這個東西R。
%           如果有需要,可以使用\includepdf來引入PDF檔案。

% 5. 論文審定同意書
%       - 這邊有附一個學校提供的公版:
%           https://aa.nycu.edu.tw/reg/regulation/
%           但假如是資訊學院的同學,可以直接在申請口試完後從系統匯出已經填好資料的這張表。
%       - 這一頁將不會出現在上傳版本中(學校的電子論文不需要這一張),
%           因此只會出現在列印輸出版中。
%           這個也是口試完才會有完整簽名的東西,所以初稿也不會出現這頁
%       - 如果有多份審定書需加入,如同時有中英文兩份之類的,
%           請先合併成一個PDF然後在這邊指定路徑
\makeApprovalPage{2-Approval/1-Approval.pdf}

% 6. 誌謝
%       依據學校的規定,可依個人意願自行決定是否撰寫,以不超過一頁為原則。
%       如果不想寫,就請將這一行註解。
%----------------------------------------------------------------------
% ACKNOWLEDGEMENT (誌謝)
%----------------------------------------------------------------------

\begin{acknowledgement}

    中文誌謝 中文誌謝 中文誌謝
    中文誌謝 中文誌謝 中文誌謝

\end{acknowledgement}


% 請不要動這一行
% 這一行代表開始編頁碼,從這一行以後開始的頁面會編羅馬數字(i, ii, iii, ...)
% 依據學校的規定,中文摘要至圖表目錄等,以 i, ii, iii...等小寫羅馬數字連續編頁。
\frontmatter

% 7. 中文摘要
%----------------------------------------------------------------------
% 中文摘要
%----------------------------------------------------------------------

% 把中文摘要寫在裡面
\begin{zhAbstract}

    烏申斯基曾經說過,學習是勞動,是充滿思想的勞動。帶著這句話,我們還要更加慎重的審視這個問題: 本人也是經過了深思熟慮,在每個日日夜夜思考這個問題。 裴斯泰洛齊曾經說過,今天應做的事沒有做,明天再早也是耽誤了。帶著這句話,我們還要更加慎重的審視這個問題: 我們不得不面對一個非常尷尬的事實,那就是, 所謂摘要,關鍵是摘要需要如何寫。 生活中,若摘要出現了,我們就不得不考慮它出現了的事實。 我認為, 一般來說, 現在,解決摘要的問題,是非常非常重要的。 所以, 要想清楚,摘要,到底是一種怎麽樣的存在。 經過上述討論我認為, 我們都知道,只要有意義,那麽就必須慎重考慮。 培根在不經意間這樣說過,深窺自己的心,而後發覺一切的奇跡在你自己。這句話語雖然很短,但令我浮想聯翩。 我們不得不面對一個非常尷尬的事實,那就是, 摘要的發生,到底需要如何做到,不摘要的發生,又會如何產生。 而這些並不是完全重要,更加重要的問題是, 一般來說, 盧梭在不經意間這樣說過,浪費時間是一樁大罪過。這不禁令我深思。 帶著這些問題,我們來審視一下摘要。 維龍在不經意間這樣說過,要成功不需要什麽特別的才能,只要把你能做的小事做得好就行了。這句話語雖然很短,但令我浮想聯翩。

    經過上述討論經過上述討論池田大作曾經說過,不要回避苦惱和困難,挺起身來向它挑戰,進而克服它。這不禁令我深思。 就我個人來說,摘要對我的意義,不能不說非常重大。

    摘要,到底應該如何實現。 既然如此, 我們不得不面對一個非常尷尬的事實,那就是, 伏爾泰曾經說過,堅持意志偉大的事業需要始終不渝的精神。帶著這句話,我們還要更加慎重的審視這個問題: 一般來講,我們都必須務必慎重的考慮考慮。 本人也是經過了深思熟慮,在每個日日夜夜思考這個問題。 培根在不經意間這樣說過,閱讀使人充實,會談使人敏捷,寫作使人精確。這句話語雖然很短,但令我浮想聯翩。 黑格爾在不經意間這樣說過,只有永遠躺在泥坑裏的人,才不會再掉進坑裏。這不禁令我深思。 每個人都不得不面對這些問題。 在面對這種問題時。

    % 這個Command會自動幫你從Config裡面設定的東東填進來
    \zhAbsKeywords
\end{zhAbstract}


% 8. 英文摘要
%----------------------------------------------------------------------
% 英文摘要
%----------------------------------------------------------------------

% 把英文摘要寫在裡面
\begin{enAbstract}

    Lorem ipsum dolor sit amet, consectetur adipiscing elit. Phasellus sed sagittis massa. Sed consectetur nisl sagittis, lacinia sapien id, mollis lectus. Nam rutrum libero erat, vel luctus mauris ornare in. Etiam a arcu eleifend, rutrum justo sed, vestibulum nibh. Pellentesque porttitor, quam ac iaculis sodales, quam est interdum eros, id scelerisque purus elit vel odio. Duis cursus varius maximus. Aliquam erat volutpat. Duis eleifend dui ut imperdiet rhoncus. Cras pulvinar, dolor vitae dapibus gravida, purus metus egestas nunc, in dignissim justo est et urna. Donec vehicula cursus congue. Etiam quis est ac nisl pretium pulvinar ut non urna. Nullam eu ultrices justo, at faucibus erat. Vestibulum id euismod lacus. Donec viverra turpis est, in maximus magna laoreet vel.

    Proin id vulputate mauris, in tempor sapien. Nulla sed felis magna. Nam molestie nunc sapien, vitae iaculis sapien dapibus ac. Duis vel risus diam. Integer egestas neque quis mi pharetra aliquet et pretium leo. Suspendisse vitae finibus nulla, ut dictum urna. Sed vel dolor at tortor elementum luctus. Vestibulum in ipsum nisi. Pellentesque ornare eros at quam eleifend, ac facilisis lacus tempor. Sed aliquet nulla bibendum turpis cursus, sit amet fermentum sapien semper. In lacinia nulla a tellus gravida, quis tincidunt sem vehicula.

    % 這個Command會自動幫你把Config裡面設定的東東填進來    
    \enAbsKeywords
\end{enAbstract}


% 9. 目錄
\maketoc

% 10. 圖目錄
%       - 如果沒有可以註解掉
\makelof

% 11. 表目錄
%       - 如果沒有可以註解掉
\makelot

% 請不要動這一行
% 這一行代表開始編頁碼,從這一行以後開始的頁面會編阿拉伯數字(1, 2, 3, ...)
% 依據學校的規定,論文第一章以至附錄,均以 1,2,3…等阿拉伯數字連續編頁。
\mainmatter

% 12. 論文正文
%       - 各章節內容,窩有在裡面附一些常用的Sample。
%           如果有需要,可隨意增減檔案。
%----------------------------------------------------------------------
% 緒論
%----------------------------------------------------------------------

\chapter{緒論\small{(如何切分段落)}}\label{chap:intro}

速再早老常小有導反則太居分的情故現實頭強維沒足們大入是決路他?前子媽解布心利轉外大他師不。決醫我代很斯早已爸家外的!建果南不展速生禮在告物事全。成下就難每飛法師事各少極保書……門雜基到解則,吃的加程銀導以樣基軍有產密文軍法示點重力西關事處了作。


\section{研究背景與動機\small(各種段落大小)}\label{sec:1-motivation}

\subsection{研究背景}

營人看歷新臺我告,學格展上元。是手去。不我非縣說業勢良我生病級做務集反車值但到也教!語命獲音所可對中傳心技重不打展百我?大生他……表有是安利達習可備質出,我環一中程難開禮也細答農說見有大機處有候著小以雜要品新因無直我成驗成意有,不起難出地像也媽不龍學出步成。海三局。例高一黨不們……教得意多裡有常!的之謝者,百絕靈而於山校,縣而銷行辦明像與叫投自朋手……才下日去大可費視省細有然旅標。通能道定體、人國速投今心經學我。戰未其月理;了拉費於來起因朋?

\subsection{研究動機}

大跑前,照石戰呢:好過沒速朋。

里保談大。黃關城多:了考找經往中道行第能臺家中一馬格,果進教……一到組立外飛道明學?理之常決養將形有一命來,從一法學有,變式出常漸亞突代早無情眾規我人麼收下地沒卻因灣,師我古今例本實火變一公變軍東下地!靈李現不動地形真育我復過到止流而任在息原影。

升計界能。一一積相來布。

\subsubsection*{動機一}

機子表月度顯們於教林靈系康開的出止心家建南生嗎說不為了我分發山感便電創的父子失能只;心此氣身心邊連可人管的的始樂細青中求社樣濟先字性球子,持文將樣月通皮世灣發子地灣民的她引謝少利已於北重名起心來亮社怎第。市始健角理全能……容地有一眼下力們點想熱保成調時制科為題外大歡近一,國當回麼眾於家感片酒事市子看也至!

\subsubsection*{動機二}

一民資在好推神,共須管苦座市古關個?他兒樣,亮覺水我時關毛人金樣高報市,一度也讀,保我出,下展民外林源也議,邊度回體活象西一。來好市系之教,在事策金;化東消驚財好打天!真音覺是來子?根源畫自合,重親電朋建全相原高如天如名機濟自報。軍方靈他年的學長。

\section{研究貢獻\small{(編號的方式)}}\label{sec:1-contribution}

\begin{enumerate}
    \item 花完造講,心城時……速政沒常,文他員這不星便說人,他的許管;家世上有案。整值八。

    \item 高文表些寫立自黃現能去……答人人痛們兒引我到向的現民爸似存算角善、心化市,人作後衣子主路。

    \item 訴點可……球的會。

    \item 留企感計多,為一家教示養河紙!

    \item 是如司她格的母都目實笑天建整打園,事點世吃勢學過們容我於線影一兒面飯新這,戰係爾說樹經的、母後畫研相大對色!到沒的後黃童感小同飛於的實道立我臺保絕!
\end{enumerate}


\section{章節概述\small{(列點的方式)}}

\begin{itemize}
    \item 在這種困難的抉擇下,本人思來想去,寢食難安。 要想清楚,章節概述,到底是一種怎麽樣的存在。 塞涅卡在不經意間這樣說過,生命如同寓言,其價值不在與長短,而在與內容。這啟發了我。
    \item 那麽, 既然如此, 斯賓諾莎在不經意間這樣說過,最大的驕傲於最大的自卑都表示心靈的最軟弱無力。這不禁令我深思。
    \item 要想清楚,章節概述,到底是一種怎麽樣的存在。 了解清楚章節概述到底是一種怎麽樣的存在,是解決一切問題的關鍵。
    \item 就我個人來說,章節概述對我的意義,不能不說非常重大。 馮學峰曾經說過,當一個人用工作去迎接光明,光明很快就會來照耀著他。帶著這句話,我們還要更加慎重的審視這個問題。
\end{itemize}
\input{5-Chapters/2-RelatedWorks}
\input{5-Chapters/3-Design}
%----------------------------------------------------------------------
% 實驗設計分析
%----------------------------------------------------------------------

\chapter{Evaluation}\label{sec:evalutaion}
\section{Experiment }\label{sec:3-experiment}
\subsection{Experiment Setup}\label{sec:3-setup}
To test the performance of the algorithm, 3 specific scenarios were tested:
1. Control scenario, when object 1 (white) and object 2 (yellow) do not crosspath.
2. One object conceals another and continues their original trajectory after departing.
3. One object conceals another and changes trajectory when departing.
\begin{figure}[!htb]
    \centering
    \begin{subfigure}{0.25\linewidth}
        \includegraphics[width=5.5cm]{Figures/scenario_1_gt.png}
        \caption{scenario 1}
        \label{subfig:scenario1gt}
    \end{subfigure}
    \hfill
    \begin{subfigure}{0.25\linewidth}
        \centering
        \includegraphics[width=5.5cm]{Figures/scenario_2_gt.png}
        \caption{scenario 2}
        \label{subfig:scenario2gt}
    \end{subfigure}
    \hfill
    \begin{subfigure}{0.25\linewidth}
        \centering
        \includegraphics[width=5.5cm]{Figures/scenario_3_gt.jpg}
        \caption{scenario 3}
        \label{subfig:scenario3gt}
    \end{subfigure}

    \caption{Ground truth of multi-object tracking experiments}
    \label{fig:ground_truth}
\end{figure}

\subsection{Challenges to Overcome}\label{sec:3-challenge}
Blind spot of both sensors.
Challenges the algorithm to correctly predict objects when data is obstructed.

\begin{figure}[!htb]
    \centering
    \begin{subfigure}{0.3\linewidth}
        \includegraphics[width=7cm]{Figures/before_conceal_radar.png}
        \caption{Raw radar data}
        \label{subfig:before_conceal_radar_fig}
    \end{subfigure}
    \hspace{0.15\textwidth}
    %\hfill
    \begin{subfigure}{0.3\linewidth}
        \includegraphics[width=7cm]{Figures/before_conceal_image.png}
        \caption{Image frame}
        \label{subfig:before_conceal_image_fig}
    \end{subfigure}

    \caption{Object 1 and object 2 before crossingpath}
    \label{fig:before_conceal_fig}
\end{figure}

From figure \ref*{fig:concealing_fig}\subref{subfig:concealing_radar_fig} can be seen that both object clusters disappear.



\begin{figure}[!htb]
    \centering
    \begin{subfigure}{0.3\linewidth}
        \includegraphics[width=7cm]{Figures/concealing_radar.png}
        \caption{Raw radar data}
        \label{subfig:concealing_radar_fig}
    \end{subfigure}
    \hspace{0.15\textwidth}
    %\hfill
    \begin{subfigure}{0.3\linewidth}
        \includegraphics[width=7cm]{Figures/concealing_image.png}
        \caption{Image frame}
        \label{subfig:concealing_image_fig}
    \end{subfigure}

    \caption{Object 1 covers object 2}
    \label{fig:concealing_fig}
\end{figure}

\begin{figure}[!htb]
    \centering
    \begin{subfigure}{0.3\linewidth}
        \includegraphics[width=7cm]{Figures/after_conceal_radar.png}
        \caption{Raw radar data}
        \label{subfig:after_conceal_radar_fig}
    \end{subfigure}
    \hspace{0.15\textwidth}
    %\hfill
    \begin{subfigure}{0.3\linewidth}
        \includegraphics[width=7cm]{Figures/after_conceal_image.png}
        \caption{Image frame}
        \label{subfig:after_conceal_image_fig}
    \end{subfigure}

    \caption{Object 1 and object 2 seperates}
    \label{fig:after_conceal_fig}
\end{figure}





\vspace*{5cm}
\subsection{Experiment Result}\label{sec:3-exp_result}
\subsubsection{Scenario 1}\label{sec:3-exp_result1}
Scenario 1 is the control when object 1 and object 2 do not crosspath.
\href{https://drive.google.com/file/d/1SL30CC6EpyI4NLOcGnfALKAP44P82u9n/view?usp=sharing}{\color{blue}{Video click me}}
\begin{figure}[!htb]
    \centering
    \begin{subfigure}{0.3\linewidth}
        \includegraphics[width=7cm]{Figures/1_before.png}
        \caption{Without fusion}
        \label{subfig:without_fusion_1}
    \end{subfigure}
    \hspace{0.15\textwidth}
    %\hfill
    \begin{subfigure}{0.3\linewidth}
        \includegraphics[width=7cm]{Figures/1_after.png}
        \caption{With fusion}
        \label{subfig:with_fusion_1}
    \end{subfigure}

    \caption{Scenario 1 multi-object tracking}
    \label{fig:scenario_result_1}
\end{figure}

\subsubsection{Scenario 2}\label{sec:3-exp_result2}
In scenario 2, object 1 crosses path with object 2 in a straight line horizontally.
After concealing object 2 from both camera and radar, both objects continue their original trajectory.
In figure \ref*{fig:scenario_result_2}\subref{subfig:without_fusion_2} can be seen that object 1 and object 2 switch places.
The algorithm is able to track and identify objects 1 and 2 correctly (figure \ref*{fig:scenario_result_2}\subref{subfig:with_fusion_2}).
\href{https://drive.google.com/file/d/1YnliV7YRzahYNpIehctzrNrf0Z0ZpIeY/view?usp=sharing}{\color{blue}{Video click me}}
\begin{figure}[!htb]
    \centering
    \begin{subfigure}{0.3\linewidth}
        \includegraphics[width=7cm]{Figures/2_before.png}
        \caption{Without fusion}
        \label{subfig:without_fusion_2}
    \end{subfigure}
    \hspace{0.15\textwidth}
    %\hfill
    \begin{subfigure}{0.3\linewidth}
        \includegraphics[width=7cm]{Figures/2_after.png}
        \caption{With fusion}
        \label{subfig:with_fusion_2}
    \end{subfigure}

    \caption{Scenario 2 multi-object tracking}
    \label{fig:scenario_result_2}
\end{figure}

\subsubsection{Scenario 3}\label{sec:3-exp_result3}
In scenario 3, object 1 covers object 2, and in the next few frames, object 2 covers object 1.
In this scenario, both objects change trajectory when departing from each other.
In figure \ref*{fig:scenario_result_3}\subref{subfig:without_fusion_3} can be seen that object 1 and object 2 switch places.
The algorithm is able to track and identify objects 1 and 2 correctly (figure \ref*{fig:scenario_result_3}\subref{subfig:with_fusion_3}).
\href{https://drive.google.com/file/d/1bd92Bu6CJeL1QIIbBlW42cAozxEcis-d/view?usp=sharing}{\color{blue}{Video click me}}
\begin{figure}[!htb]
    \centering
    \begin{subfigure}{0.3\linewidth}
        \includegraphics[width=7cm]{Figures/3_before.png}
        \caption{Without fusion}
        \label{subfig:without_fusion_3}
    \end{subfigure}
    \hspace{0.15\textwidth}
    %\hfill
    \begin{subfigure}{0.3\linewidth}
        \includegraphics[width=7cm]{Figures/3_after.png}
        \caption{With fusion}
        \label{subfig:with_fusion_3}
    \end{subfigure}

    \caption{Scenario 3 multi-object tracking}
    \label{fig:scenario_result_3}
\end{figure}
%----------------------------------------------------------------------
% 結論與未來展望
%----------------------------------------------------------------------

\chapter{Conclusion}\label{chap:conclusion}
result of my research

\section{Further Improvement}
Image to radar mapping assumes that both sensors are perfectly parallel
The algorithm cannot determine true center of an object, 
it just assumes the average center of the cluster.
Although there is intensity in data input, it is not used.

% 13. 參考文獻
\makeBib

% 14. 附錄
%       - ㄅ歉,窩的論文沒有寫到附錄的部分,所以窩不是很確定這邊的格式該怎麼設。
%           但學校的規定又偏鬆散。所以窩就提供了一個窩的設計給各位參考一下
%       - 底下的兩個檔案裡面有sample說要怎麼使用,我是設計了一個新的environment,
%           所以如果覺得很醜或是不想用的話,就直接去用其他套件就好了,應該不會有衝突。吧。
%----------------------------------------------------------------------
% APPENDIX (附錄頁)
%----------------------------------------------------------------------

\begin{Appx}{這是第一個附錄}
    \section{這是附錄裡面的節}
    \subsection{這是一節}
    \subsubsection{這是一點}
    附錄 附錄 附錄 附錄 附錄 附錄 附錄 附錄 附錄 附錄 附錄 附錄 附錄 附錄
\end{Appx}

% \begin{Appx}[Appx-1]
%     附錄 附錄 附錄 附錄 附錄 附錄 附錄 附錄 附錄 附錄 附錄 附錄 附錄 附錄
% \end{Appx}

% HINT: Write down your contents here
%----------------------------------------------------------------------
% APPENDIX (附錄頁)
%----------------------------------------------------------------------

\begin{Appx}{這是另外一個很長篇大論的附錄}
    生活中,若這是另外一個很長篇大論的附錄出現了,我們就不得不考慮它出現了的事實。 叔本華曾經說過,普通人只想到如何度過時間,有才能的人設法利用時間。這句話語雖然很短,但令我浮想聯翩。 這是另外一個很長篇大論的附錄,到底應該如何實現。 一般來說, 問題的關鍵究竟為何? 問題的關鍵究竟為何? 了解清楚這是另外一個很長篇大論的附錄到底是一種怎麽樣的存在,是解決一切問題的關鍵。 在這種困難的抉擇下,本人思來想去,寢食難安。 雷鋒曾經說過,自己活著,就是為了使別人過得更美好。這不禁令我深思。 我們不得不面對一個非常尷尬的事實,那就是, 富蘭克林在不經意間這樣說過,你熱愛生命嗎?那麽別浪費時間,因為時間是組成生命的材料。這句話語雖然很短,但令我浮想聯翩。 一般來說, 吉格·金克拉在不經意間這樣說過,如果你能做夢,你就能實現它。這不禁令我深思。 我們都知道,只要有意義,那麽就必須慎重考慮。 一般來說, 我們不得不面對一個非常尷尬的事實,那就是, 生活中,若這是另外一個很長篇大論的附錄出現了,我們就不得不考慮它出現了的事實。 而這些並不是完全重要,更加重要的問題是, 總結的來說, 笛卡兒在不經意間這樣說過,讀一切好書,就是和許多高尚的人談話。我希望諸位也能好好地體會這句話。 而這些並不是完全重要,更加重要的問題是, 既然如何, 這是另外一個很長篇大論的附錄,發生了會如何,不發生又會如何。 了解清楚這是另外一個很長篇大論的附錄到底是一種怎麽樣的存在,是解決一切問題的關鍵。 要想清楚,這是另外一個很長篇大論的附錄,到底是一種怎麽樣的存在。 我認為, 這是另外一個很長篇大論的附錄的發生,到底需要如何做到,不這是另外一個很長篇大論的附錄的發生,又會如何產生。 這是另外一個很長篇大論的附錄,到底應該如何實現。 美華納在不經意間這樣說過,勿問成功的秘訣為何,且盡全力做你應該做的事吧。這句話語雖然很短,但令我浮想聯翩。

    在這種困難的抉擇下,本人思來想去,寢食難安。 我們不得不面對一個非常尷尬的事實,那就是, 了解清楚這是另外一個很長篇大論的附錄到底是一種怎麽樣的存在,是解決一切問題的關鍵。 這種事實對本人來說意義重大,相信對這個世界也是有一定意義的。 既然如何, 馬克思曾經說過,一切節省,歸根到底都歸結為時間的節省。這句話語雖然很短,但令我浮想聯翩。 每個人都不得不面對這些問題。 在面對這種問題時, 歌德曾經說過,沒有人事先了解自己到底有多大的力量,直到他試過以後才知道。我希望諸位也能好好地體會這句話。 生活中,若這是另外一個很長篇大論的附錄出現了,我們就不得不考慮它出現了的事實。 在這種困難的抉擇下,本人思來想去,寢食難安。 生活中,若這是另外一個很長篇大論的附錄出現了,我們就不得不考慮它出現了的事實。 史美爾斯曾經說過,書籍把我們引入最美好的社會,使我們認識各個時代的偉大智者。這不禁令我深思。 在這種困難的抉擇下,本人思來想去,寢食難安。 我認為, 問題的關鍵究竟為何? 這種事實對本人來說意義重大,相信對這個世界也是有一定意義的。 生活中,若這是另外一個很長篇大論的附錄出現了,我們就不得不考慮它出現了的事實。 那麽, 那麽, 我們一般認為,抓住了問題的關鍵,其他一切則會迎刃而解。 我們一般認為,抓住了問題的關鍵,其他一切則會迎刃而解。 問題的關鍵究竟為何? 這種事實對本人來說意義重大,相信對這個世界也是有一定意義的。 我認為, 要想清楚,這是另外一個很長篇大論的附錄,到底是一種怎麽樣的存在。 經過上述討論我認為, 這是另外一個很長篇大論的附錄,發生了會如何,不發生又會如何。 總結的來說, 在這種困難的抉擇下,本人思來想去,寢食難安。 每個人都不得不面對這些問題。 在面對這種問題時, 這是另外一個很長篇大論的附錄,到底應該如何實現。 而這些並不是完全重要,更加重要的問題是, 我們一般認為,抓住了問題的關鍵,其他一切則會迎刃而解。

    這種事實對本人來說意義重大,相信對這個世界也是有一定意義的。 笛卡兒在不經意間這樣說過,讀一切好書,就是和許多高尚的人談話。這不禁令我深思。 我認為, 既然如何, 我們都知道,只要有意義,那麽就必須慎重考慮。 一般來講,我們都必須務必慎重的考慮考慮。 這是另外一個很長篇大論的附錄的發生,到底需要如何做到,不這是另外一個很長篇大論的附錄的發生,又會如何產生。 既然如何, 我們都知道,只要有意義,那麽就必須慎重考慮。 這是另外一個很長篇大論的附錄,到底應該如何實現。 本人也是經過了深思熟慮,在每個日日夜夜思考這個問題。 每個人都不得不面對這些問題。 在面對這種問題時, 就我個人來說,這是另外一個很長篇大論的附錄對我的意義,不能不說非常重大。 我們一般認為,抓住了問題的關鍵,其他一切則會迎刃而解。 在這種困難的抉擇下,本人思來想去,寢食難安。 這種事實對本人來說意義重大,相信對這個世界也是有一定意義的。 既然如何, 就我個人來說,這是另外一個很長篇大論的附錄對我的意義,不能不說非常重大。 而這些並不是完全重要,更加重要的問題是, 要想清楚,這是另外一個很長篇大論的附錄,到底是一種怎麽樣的存在。 奧普拉·溫弗瑞在不經意間這樣說過,你相信什麽,你就成為什麽樣的人。這不禁令我深思。 本人也是經過了深思熟慮,在每個日日夜夜思考這個問題。 既然如何, 生活中,若這是另外一個很長篇大論的附錄出現了,我們就不得不考慮它出現了的事實。 我認為, 我認為, 這種事實對本人來說意義重大,相信對這個世界也是有一定意義的。 問題的關鍵究竟為何? 那麽, 在這種困難的抉擇下,本人思來想去,寢食難安。 一般來講,我們都必須務必慎重的考慮考慮。 帶著這些問題,我們來審視一下這是另外一個很長篇大論的附錄。 本人也是經過了深思熟慮,在每個日日夜夜思考這個問題。 這種事實對本人來說意義重大,相信對這個世界也是有一定意義的。

    這是另外一個很長篇大論的附錄,到底應該如何實現。 每個人都不得不面對這些問題。 在面對這種問題時, 了解清楚這是另外一個很長篇大論的附錄到底是一種怎麽樣的存在,是解決一切問題的關鍵。 現在,解決這是另外一個很長篇大論的附錄的問題,是非常非常重要的。 所以, 了解清楚這是另外一個很長篇大論的附錄到底是一種怎麽樣的存在,是解決一切問題的關鍵。 我們不得不面對一個非常尷尬的事實,那就是, 既然如此, 總結的來說, 一般來講,我們都必須務必慎重的考慮考慮。 每個人都不得不面對這些問題。 在面對這種問題時, 馬克思在不經意間這樣說過,一切節省,歸根到底都歸結為時間的節省。我希望諸位也能好好地體會這句話。 帶著這些問題,我們來審視一下這是另外一個很長篇大論的附錄。 問題的關鍵究竟為何? 經過上述討論生活中,若這是另外一個很長篇大論的附錄出現了,我們就不得不考慮它出現了的事實。 既然如此。
\end{Appx}
% HINT: Write down your contents here

% 15. 封底
%       - 這一行就是窩寫開心的,Yeah。

% 論文結束喇
\end{document}
