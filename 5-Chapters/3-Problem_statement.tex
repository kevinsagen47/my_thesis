%%%%%%%%%%%%%%%%%%%%%%%%%%%%%%%%%%%%%%% PROBLEM FORMULATION OR STATEMENT %%%%%%%%%%%%%%%%%%%%%%%%%%%%%%


\chapter{Problem Statement}\label{chap:Problem_statement}

\section{Fusion Algorithm}\label{sec:2-bayes_fusion}
Fusing measurements from two heterogeneous sensors that have few cross-correlations can be challenging. 
To take inputs from two different sensors and give them weight, we use Bayes Fusion in this thesis.
Bayes fusion takes the noises of both sensors to determine their reliability at a given point of the measurement.
Thus giving us a reliable way to give scores of trust to each sensor.

The coordinates that undergo fusion from both radar and image sensors are the horizontal coordinates,
specifically the radar's azimuth and the image's "u" coordinate. 
This constraint arises from the fact that these are the only coordinates where both sensors provide measurements, 
as illustrated in Figure \ref{fig:trade_off_and_plane}\subref{subfig:cam_radar_sub}.
It's important to note that radar's elevation measurement resolution is limited and subject to noise. 
Additionally, the image sensor does not provide depth information.

If X is the real position of the object, 
then Bayes' theorem predicts that the probability of the fused position is shown in equation \ref{equ:bayes1} \cite{10.1007/978-981-16-2248-9_32}.

\begin{equation}\label{equ:bayes1}
    P_{prob}(\frac{P}{X})=
    \frac
    {e \frac{−(P−X)^T R^{−1}(P−X)}{2}}
    {2 \pi R^(0.5)}
\end{equation}

Applying Bayes' fusion, the value of the measured measurements is provided by equation \ref{equ:bayes2}.

\begingroup
\large
\begin{equation}\label{equ:bayes2}
\theta_{bayes}=\frac
{\frac{\theta_{radar}}{R_{radar}}+\frac{\theta_{cam}} {R_{cam}}}
{\frac{1}{R_{radar}}+\frac{1}{R_{cam}}}
\end{equation}
\endgroup

where
\begin{align*}
    \theta_{bayes} &= \text{fused position}\\
    \theta_{radar} &= \text{radar azimuth angle}\\
    \theta_{cam} &= \text{camera azimuth angle}\\
    R_{radar} &= \text{radar covariance}\\
    R_{cam} &= \text{camera covariance}
\end{align*}
\begin{equation}\label{equ:bayes4}
    \frac{1}{R}=\frac{1}{R_1}+\frac{1}{R_2}
\end{equation}
\begin{equation}\label{equ:2-radar_R}
    \mathbf{R}_{radar} = 
        \sigma_{radar_x}^2 
\end{equation}
\begin{equation}\label{equ:2-R_cam}
    \mathbf{R}_{cam} = 
        \sigma_{cam_u}^2
\end{equation}


\section{Motion Model}\label{sec:2-kalman_filter}
%To predict the movement model of our subjects, we use a Constant Acceleration model of Kalman filter.

\subsection{Predict}\label{sec:2-predict}
The state matrix used in this Kalman Filter \cite{kalman} is from the single sensor maneuvering tracking, with constant velocity.
It has four elements and is defined with position and velocity, which projects onto the x-axis and y-axis.
State vector also includes the width and the height of the object\cite{4732695}.
\begin{equation}\label{equ:state_eq}
    \mathbf{x} = 
        \begin{bmatrix} 
        p \\ 
        v \\
        w \\
        h \\

        \end{bmatrix} = 
        \begin{bmatrix} 
        p_x \\ 
        p_y \\ 
        v_x \\ 
        v_y \\
        w \\
        h \\
        \end{bmatrix}
\end{equation}
where
\begin{align*}
    p_x &=\text{position in x-axis}\\
    p_y &=\text{position in y-axis}\\
    v_x &=\text{velocity in x-axis}\\
    v_y &=\text{velocity in y-axis}\\
    w &=\text{width}\\%\footnote{\label{note1}Obtained from equation} \\
    h &=\text{height}\\
\end{align*}
%\footnotetext{Obtained from equation \ref{equ:2_cam_height}}
Transition Matrix is expressed as follows:
\begin{equation}\label{equ:transition_matrix_H}
    \mathbf{F} = 
    \begin{bmatrix}
        1 & 0 & 1 & 0 & 0 & 0\\
        0 & 1 & 0 & 1 & 0 & 0\\
        0 & 0 & 0 & 0 & 0 & 0\\
        0 & 0 & 0 & 0 & 0 & 0\\
        0 & 0 & 0 & 0 & 1 & 0\\
        0 & 0 & 0 & 0 & 0 & 1\\
      \end{bmatrix}
\end{equation}
Thus state vector is updated as below:
\begin{equation}\label{equ:predict_eq}
    \mathbf{x}_k=\mathbf{F}_k\mathbf{x}_{k-1}+\mathbf{w}
\end{equation}

Error covariance update:
\begin{equation}\label{equ:error_covariance}
    \mathbf{P}_k=\mathbf{F}_k \mathbf{P}_{k-1} \mathbf{F}_k^T+\mathbf{Q}_k
\end{equation}

%\newpage
\subsection{Update}\label{equ:2_update}

Both radar and camera undergo the same update procedure within the Kalman filter framework. 
Nevertheless, variations exist in their measurement data, 
leading to distinct measurement matrices and noise covariance. 
Radar measurements encompass two parameters: azimuth and range, 
both presented in polar coordinates. 
These radar coordinates are derived from the centroids of the k-d tree clusters.
Meanwhile, the width and height measurements are taken from equation \ref{equ:2_cam_width} and equation \ref{equ:2_cam_height}.

\begin{equation}
    \mathbf{z}=
    \begin{bmatrix}
        \rho \\ 
        \theta_{bayes}\\
        w \\
        h \\
    \end{bmatrix}
\end{equation}


After the measurement matrix $\mathbf{z}_k$ is obtained, 
it is subtracted from the previously predicted values by the Kalman Filter. 
\begin{equation}
    \mathbf{y}_{k}=\mathbf{z}-\mathbf{H}_k \mathbf{x}_k
\end{equation}
Subsequently, the matrix $ \mathbf{G}_k $ is calculated to determine the trustworthiness of the current measurement, taking into account measurement noise. 
\begin{equation}
    \mathbf{G}_k = \frac{\mathbf{P}_k \mathbf{H}_k^T}{\mathbf{H}_k\mathbf{P}_k\mathbf{H}_k^T + \mathbf{R}_k}
\end{equation}
Finally, the current state estimation is updated based on this adjusted measurement.
\begin{equation}
    \mathbf{P}_k = (\mathbf{I} - \mathbf{G}_k\mathbf{H})\mathbf{P}_k
\end{equation}

%\newpage
\newpage
\subsection{Non-linearity}\label{equ:2_non_linear}
Since measurements and Bayes fusion results are obtained and calculated in polar coordinates, 
the application of the Extended Kalman Filter becomes crucial to linearize these results into Cartesian coordinates. 
$h(\mathbf{x}_k)$ matrix represents conversion of polar coordinate to the cartesian coordinate, which is defined as follows:
\begin{equation}
    h(\mathbf{x}_k)=
    \begin{bmatrix}
        \rho \\ 
        \theta_{bayes}\\
        w \\
        h \\
    \end{bmatrix}=
    \begin{bmatrix}
    \sqrt{p_x^2+p_y^2}\\
    \tan^{-1}(\frac{p_x}{p_y})\\
    w \\
    h \\
    \end{bmatrix}
\end{equation}

Because the solution comprises higher-dimensional components than the equations, 
a first-order Jacobian matrix is employed on $\mathbf{H}$: 
\begingroup
        \large
        \begin{equation}
            \mathbf{H}=
            \begin{bmatrix}
                \frac{\partial h(\mathbf{x}_k)}{\partial \mathbf{x}_k}
            \end{bmatrix}=
            \begin{bmatrix}
                \frac{\partial \rho}{\partial p_x} & \frac{\partial \rho}{\partial p_y}
                & \frac{\partial \rho}{\partial v_x}& \frac{\partial \rho}{\partial v_y} 
                & \frac{\partial \rho}{\partial w}& \frac{\partial \rho}{\partial h}\\
        
                \frac{\partial \theta}{\partial p_x} & \frac{\partial \theta}{\partial p_y} 
                & \frac{\partial \theta}{\partial v_x}& \frac{\partial \theta}{\partial v_y} 
                & \frac{\partial \theta}{\partial w}& \frac{\partial \theta}{\partial h} \\

                \frac{\partial w}{\partial p_x} & \frac{\partial w}{\partial p_y} 
                & \frac{\partial w}{\partial v_x}& \frac{\partial w}{\partial v_y} 
                & \frac{\partial w}{\partial w}& \frac{\partial w}{\partial h} \\

                \frac{\partial h}{\partial p_x} & \frac{\partial h}{\partial p_y} 
                & \frac{\partial h}{\partial v_x}& \frac{\partial h}{\partial v_y} 
                & \frac{\partial h}{\partial w}& \frac{\partial h}{\partial h} \\
            \end{bmatrix}
        \end{equation}
    Thus giving the transition matrix as follows:    
    \begin{equation}
            \mathbf{H}=
            \begin{bmatrix}
                \frac{p_x}{\sqrt{p_x^2+p_y^2}} & \frac{p_y}{\sqrt{p_x^2+p_y^2}} & 0 & 0 & 0 & 0\\
                -\frac{p_y}{p_x^2+p_y^2} & \frac{p_x}{p_x^2+p_y^2} & 0 & 0 & 0 & 0\\
                0 & 0 & 0 & 0 & 1 & 0\\
                0 & 0 & 0 & 0 & 0 & 1
            \end{bmatrix}
        \end{equation}
    \endgroup