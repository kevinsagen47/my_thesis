%----------------------------------------------------------------------
% 實驗設計分析
%----------------------------------------------------------------------

\chapter{實驗設計分析\small{(如何插入表格和圖片)}}\label{sec:evalutaion}

\section{實驗設計}

圖\ref{fig:figexample}。圖\ref{fig:figexample2}。圖\ref{fig:figexample3}。表\ref{tab:tabexample}。表\ref{tab:tabexample2}。表\ref{tab:tabexample3}。表\ref{tab:tabexample4}。表\ref{tab:tabexample5}。話雖如此,需要考慮周詳實驗設計的影響及因應對策。鄒韜奮曾講過,友誼是天地間最可寶貴的東西,深摯的友誼是人生最大的一種安慰。這啟發了我。我們一般認為,抓住了問題的關鍵,其他一切則會迎刃而解。對於一般人來說,實驗設計究竟象徵著什麼呢?儘管實驗設計看似不顯眼,卻佔據了我的腦海。茅盾講過一段深奧的話,過去的,讓它過去,永遠不要回顧; 未來的,等來了時再說,不要空想; 我們只抓住了現在,用我們現在的理想,做我們所應該做的。這段話讓我所有的疑惑頓時豁然開朗。一般來講,我們都必須務必慎重的考慮考慮。

\begin{table}[ht]
    \centering
    \renewcommand{\arraystretch}{1.2}

    \begin{tabular}{ c | l l }

        \diagbox[innerwidth=6em,trim=l]{$\alpha$}{$\beta$} & \multicolumn{2}{c}{這一格透過\textbackslash multicolumn可以跨2欄並且置中}                          \\
        \hline\hline
        \multirow{2}{*}{跨列A}                             & 這邊2欄的內容會靠左對齊                                                   & 而且不會平均分配       \\\cline{2-3}
                                                           & 要平均分配的話                                                            & 請看下面的另外一個範例 \\\hline
        \multirow{5}{*}{跨列B}                             & $\alpha  $                                                                & $\beta  $              \\\cline{2-3}
                                                           & $\gamma  $                                                                & $\delta  $             \\\cline{2-3}
                                                           & $\epsilon  $                                                              & $\varepsilon  $        \\\cline{2-3}
                                                           & $\zeta  $                                                                 & $\eta  $               \\\cline{2-3}
                                                           & $\theta  $                                                                & $\vartheta $           \\\hline
    \end{tabular}

    \renewcommand{\arraystretch}{1}

    \caption{範例表格A}
    \label{tab:tabexample}
\end{table}

\begin{table}[ht]
    \centering
    \renewcommand{\arraystretch}{1.2}

    \begin{tabularx}{\textwidth}{l|l|X}
        \hline
        $\iota $   & $\kappa $ & 這一個欄位會自動被撐開到滿足表格總寬度 \\
        \hline\hline
        $\lambda $ & $\mu $    & $\nu $                                 \\
        $\xi $     & $o $      & $\pi $                                 \\
        \hline
    \end{tabularx}

    \caption{使用tabularx可以指定表格總寬度,搭配它內建的Column Type "X",可以撐開格子。}
    \label{tab:tabexample2}
\end{table}

\begin{table}[ht]
    \centering
    \renewcommand{\arraystretch}{1.2}

    \begin{tabular}{ c | c | C{10em}}
        $\varpi $   & $\rho  $      & 這是一個內容有點多的格子,它會自動換行而且置中 \\ \hline\hline
        $\varrho  $ & $\sigma  $    & $\varsigma  $                                  \\\hline
        $\tau  $    & $\upsilon   $ & $\phi   $                                      \\\hline
        $\varphi $  & $\chi   $     & $\psi   $                                      \\\hline
    \end{tabular}

    \renewcommand{\arraystretch}{1}

    \caption{使用tabular,然後使用模板提供的New Column Type "C",可以指定格子寬度然後自動換行並且置中}
    \label{tab:tabexample3}
\end{table}

\begin{figure}[hpbt]
    \centering
    \includegraphics[width=\textwidth]{Figures/computer_science.jpg}
    \caption{範例圖片A}
    \label{fig:figexample}
\end{figure}

\begin{table}[ht]
    \centering
    \renewcommand{\arraystretch}{1.2}

    \begin{tabularx}{\textwidth}{ c *{6}{|Y} }
        \multirow{3}{*}{$\omega  $} & \multicolumn{6}{c}{這邊底下的六個欄位會好好的平均分好}                                                                                                           \\\cline{2-7}
                                    & \multicolumn{3}{c|}{$A  $}                             & \multicolumn{3}{c}{$B  $}                                                                               \\\cline{2-7}
                                    & $\Gamma $                                              & $\varGamma $                       & $\Delta  $ & $\varDelta $              & $E $       & $Z $         \\
        \hline\hline
        $H  $                       & $\Theta  $                                             & \multicolumn{2}{c|}{$\varTheta  $} & $I  $      & \multicolumn{2}{c}{$K  $}                             \\\hline
        $\Lambda  $                 & $\varLambda  $                                         & $M  $                              & $N  $      & $\Xi  $                   & $\varXi  $ & $O  $        \\
        \hline\hline
        $\Pi  $                     & $\varPi $                                              & $P  $                              & $\Sigma  $ & $\varSigma  $             & $T  $      & $\Upsilon  $ \\\hline
    \end{tabularx}
    \renewcommand{\arraystretch}{1}

    \caption{使用tabularx搭配模板提供的New Column Type "Y"可以平均分配格子寬度。}
    \label{tab:tabexample4}
\end{table}

\begin{table}[ht]
    \centering
    \renewcommand{\arraystretch}{1.2}

    \begin{adjustbox}{center}
        \begin{tabular*}{1.1\textwidth}{  *6{ c |} @{\extracolsep{\fill}} cccc }
            \multirow{2}{*}{$\varUpsilon  $}     & \multicolumn{9}{c}{這個表格超級超級寬} \\\cline{2-10}
            & $\Phi  $       & $\varPhi  $        & $X  $      & $\Psi  $       & $\varPsi  $        & $\Omega  $ & $\varOmega  $       & $\aleph  $        & $\beth  $ \\
            \hline\hline
            $\daleth  $                                       &        $\gimel  $      & $\vert  $                    & $\Vert $                &      $\langle  $               & $\rangle  $            & $\lfloor  $            & $\rfloor  $                    & $\lceil  $             & $\rceil  $              \\\hline
            $\Uparrow  $                                   &      $\uparrow  $          & $\Downarrow  $                 & $\downarrow  $                   &    $\llcorner $                 & $\lrcorner  $              & $\ulcorner  $              &  $\urcorner  $                   & $\ast  $              & $\star  $              \\\hline
            $\cdot  $                                  &   $\bullet  $             & $\circ  $                     &$\bigcirc  $                   &    $\diamond  $                & $\times  $              & $\div  $              &   $\centerdot  $                & $\circledast  $              & $\circledcirc  $              \\\hline
            $\circledcirc  $                                         &    $\circleddash  $           & $\dotplus  $                    & $\divideontimes  $                  &   $\pm  $                 & $\mp  $              & $\amalg  $              &    $\odot  $                & $\ominus  $              & $\oplus  $              \\\hline
            $\oslash  $                                 &     $\otimes  $                & $\wr  $                     & $\Box  $                   &        $\boxplus  $           &$\boxminus  $             & $\boxtimes  $              &            $\boxdot  $      & $\square  $            & $\cap  $              \\
            \hline\hline
            $\cup  $                                   &        $\uplus  $                  & $\sqcap  $                    & $\sqcup  $                   &       $\wedge  $                & $\vee  $            & $\dagger  $              &      $\ddagger  $               & $\barwedge  $             & $\veebar  $             \\\hline
        \end{tabular*}
    \end{adjustbox}

    \renewcommand{\arraystretch}{1}

    \caption{這是一個會炸出邊界的超寬表格,使用adjustbox讓表格還是維持置中。}
    \label{tab:tabexample5}
\end{table}

\begin{figure}[hbpt]
    \centering
    \begin{subfigure}{0.45\linewidth}
        \includegraphics[width=\textwidth]{Figures/computer_science.jpg}
        \caption{左邊放個圖片}
    \end{subfigure}
    \hfill
    \begin{subfigure}{0.45\linewidth}
        \includegraphics[width=\textwidth]{Figures/computer_science.jpg}
        \caption{右邊放個圖片}
    \end{subfigure}
    \caption{2張圖擺在一起}
    \label{fig:figexample2}
\end{figure}

\begin{figure}[hbpt]
    \centering
    \begin{subfigure}{0.4\linewidth}
        \includegraphics[width=\textwidth]{Figures/computer_science.jpg}
        \caption{左邊放個圖片}
    \end{subfigure}
    \hfill
    \begin{subfigure}{0.48\linewidth}
        \centering
        \begin{tabular}{c | c }
            $\mathcal{A} \mathcal{B}  $                   & $\mathbb{A} \mathbb{B}  $                                               \\
            \hline \hline
            $\mathfrak{A} \mathfrak{B}  $                 & $\mathsf{A} \mathsf{B}  $                                               \\
            $\mathbf{A} \mathbf{B}  $                     & $\clubsuit \diamondsuit \heartsuit \spadesuit  $                        \\
            $ \looparrowleft \looparrowright \Lsh \Rsh  $ & $\curvearrowleft \curvearrowright \circlearrowleft \circlearrowright  $ \\
        \end{tabular}
        \caption{右邊放個表格}
    \end{subfigure}
    \caption{圖表擺在一起}
    \label{fig:figexample3}
\end{figure}