%----------------------------------------------------------------------
% 實驗設計分析
%----------------------------------------------------------------------

\chapter{Result}\label{sec:evalutaion}

\section{實驗設計}

圖\ref{fig:figexample}。圖\ref{fig:figexample2}。圖\ref{fig:figexample3}。
表\ref{tab:tabexample}。表\ref{tab:tabexample2}。表\ref{tab:tabexample3}。
表\ref{tab:tabexample4}。表\ref{tab:tabexample5}。話雖如此

\begin{table}[ht]
    \centering
    \renewcommand{\arraystretch}{1.2}

    \begin{tabular}{ c | l l }

        \diagbox[innerwidth=6em,trim=l]{$\alpha$}{$\beta$} & \multicolumn{2}{c}{這一格透過\textbackslash multicolumn可以跨2欄並且置中}                   \\
        \hline\hline
        \multirow{2}{*}{跨列A}                               & 這邊2欄的內容會靠左對齊                                                 & 而且不會平均分配        \\\cline{2-3}
                                                           & 要平均分配的話                                                      & 請看下面的另外一個範例     \\\hline
        \multirow{5}{*}{跨列B}                               & $\alpha  $                                                   & $\beta  $       \\\cline{2-3}
                                                           & $\gamma  $                                                   & $\delta  $      \\\cline{2-3}
                                                           & $\epsilon  $                                                 & $\varepsilon  $ \\\cline{2-3}
                                                           & $\zeta  $                                                    & $\eta  $        \\\cline{2-3}
                                                           & $\theta  $                                                   & $\vartheta $    \\\hline
    \end{tabular}

    \renewcommand{\arraystretch}{1}

    \caption{範例表格A}
    \label{tab:tabexample}
\end{table}


\begin{figure}[hbpt]
    \centering
    \begin{subfigure}{0.45\linewidth}
        \includegraphics[width=\textwidth]{Figures/computer_science.jpg}
        \caption{左邊放個圖片}
    \end{subfigure}
    \hfill
    \begin{subfigure}{0.45\linewidth}
        \includegraphics[width=\textwidth]{Figures/computer_science.jpg}
        \caption{右邊放個圖片}
    \end{subfigure}
    \caption{2張圖擺在一起}
    \label{fig:figexample2}
\end{figure}

\begin{figure}[hbpt]
    \centering
    \begin{subfigure}{0.4\linewidth}
        \includegraphics[width=\textwidth]{Figures/computer_science.jpg}
        \caption{左邊放個圖片}
    \end{subfigure}
    \hfill
    \begin{subfigure}{0.48\linewidth}
        \centering
        \begin{tabular}{c | c }
            $\mathcal{A} \mathcal{B}  $                   & $\mathbb{A} \mathbb{B}  $                                               \\
            \hline \hline
            $\mathfrak{A} \mathfrak{B}  $                 & $\mathsf{A} \mathsf{B}  $                                               \\
            $\mathbf{A} \mathbf{B}  $                     & $\clubsuit \diamondsuit \heartsuit \spadesuit  $                        \\
            $ \looparrowleft \looparrowright \Lsh \Rsh  $ & $\curvearrowleft \curvearrowright \circlearrowleft \circlearrowright  $ \\
        \end{tabular}
        \caption{右邊放個表格}
    \end{subfigure}
    \caption{圖表擺在一起}
    \label{fig:figexample3}
\end{figure}