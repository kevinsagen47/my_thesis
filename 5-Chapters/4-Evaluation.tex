%----------------------------------------------------------------------
% 實驗設計分析
%----------------------------------------------------------------------

\chapter{實驗設計分析\small{(如何插入表格和圖片)}}\label{sec:evalutaion}

\section{實驗設計}

圖\ref{fig:figexample}。圖\ref{fig:figexample2}。圖\ref{fig:figexample3}。表\ref{tab:tabexample}。表\ref{tab:tabexample2}。表\ref{tab:tabexample3}。表\ref{tab:tabexample4}。表\ref{tab:tabexample5}。話雖如此,需要考慮周詳實驗設計的影響及因應對策。鄒韜奮曾講過,友誼是天地間最可寶貴的東西,深摯的友誼是人生最大的一種安慰。這啟發了我。我們一般認為,抓住了問題的關鍵,其他一切則會迎刃而解。對於一般人來說,實驗設計究竟象徵著什麼呢?儘管實驗設計看似不顯眼,卻佔據了我的腦海。茅盾講過一段深奧的話,過去的,讓它過去,永遠不要回顧; 未來的,等來了時再說,不要空想; 我們只抓住了現在,用我們現在的理想,做我們所應該做的。這段話讓我所有的疑惑頓時豁然開朗。一般來講,我們都必須務必慎重的考慮考慮。

\begin{table}[ht]
    \centering
    \renewcommand{\arraystretch}{1.2}

    \begin{tabular}{ c | l l }

        \diagbox[innerwidth=6em,trim=l]{$\alpha$}{$\beta$} & \multicolumn{2}{c}{這一格透過\textbackslash multicolumn可以跨2欄並且置中}                          \\
        \hline\hline
        \multirow{2}{*}{跨列A}                             & 這邊2欄的內容會靠左對齊                                                   & 而且不會平均分配       \\\cline{2-3}
                                                           & 要平均分配的話                                                            & 請看下面的另外一個範例 \\\hline
        \multirow{5}{*}{跨列B}                             & $\varepsilon $                                                            & $\zeta $               \\\cline{2-3}
                                                           & $\eta $                                                                   & $\theta $              \\\cline{2-3}
                                                           & $\vartheta $                                                              & $\iota $               \\\cline{2-3}
                                                           & $\kappa $                                                                 & $\lambda $             \\\cline{2-3}
                                                           & $\mu $                                                                    & $\nu $                 \\\hline
    \end{tabular}

    \renewcommand{\arraystretch}{1}

    \caption{範例表格A}
    \label{tab:tabexample}
\end{table}

\begin{table}[ht]
    \centering
    \renewcommand{\arraystretch}{1.2}

    \begin{tabularx}{\textwidth}{l|l|X}
        \hline
        $\mathfrak{E}$ & $\mathfrak{F}$ & 這一個欄位會自動被撐開到滿足表格總寬度 \\
        \hline\hline
        $\mathfrak{H}$ & $\mathfrak{I}$ & $\mathfrak{P}$                         \\
        $\mathfrak{Z}$ & $\mathfrak{W}$ & $\mathfrak{T}$                         \\
        \hline
    \end{tabularx}

    \caption{使用tabularx可以指定表格總寬度,搭配它內建的Column Type "X",可以撐開格子。}
    \label{tab:tabexample2}
\end{table}

\begin{table}[ht]
    \centering
    \renewcommand{\arraystretch}{1.2}

    \begin{tabular}{ c | c | C{10em}}
        $\alpha$        & $\zeta $     & 這是一個內容有點多的格子,它會自動換行而且置中 \\ \hline\hline
        $\mathfrak{A} $ & $\vartheta $ & $\iota $                                       \\\hline
        $\mathfrak{Y} $ & $\mu  $      & $\nu  $                                        \\\hline
        $\mathfrak{P} $ & $\pi  $      & $\varpi  $                                     \\\hline
    \end{tabular}

    \renewcommand{\arraystretch}{1}

    \caption{使用tabular,然後使用模板提供的New Column Type "C",可以指定格子寬度然後自動換行並且置中}
    \label{tab:tabexample3}
\end{table}

\begin{figure}[hpbt]
    \centering
    \includegraphics[width=\textwidth]{Figures/computer_science.jpg}
    \caption{範例圖片A}
    \label{fig:figexample}
\end{figure}

\begin{table}[ht]
    \centering
    \renewcommand{\arraystretch}{1.2}

    \begin{tabularx}{\textwidth}{ c *{6}{|Y} }
        \multirow{3}{*}{$\varsigma $} & \multicolumn{6}{c}{這邊底下的六個欄位會好好的平均分好}                                                                                                      \\\cline{2-7}
                                      & \multicolumn{3}{c|}{$\upsilon $}                       & \multicolumn{3}{c}{$\phi $}                                                                        \\\cline{2-7}
                                      & $\alpha$                                               & $\beta$                        & $\gamma $ & $\alpha$                       & $\beta$ & $\gamma$   \\
        \hline\hline
        $\delta $                     & $\zeta $                                               & \multicolumn{2}{c|}{$\theta $} & $\iota $  & \multicolumn{2}{c}{$\lambda $}                        \\\hline
        $\epsilon $                   & $\eta $                                                & $\vartheta $                   & $\kappa $ & $\mu $                         & $\nu $  & $\xi $     \\
        \hline\hline
        $\varepsilon $                & -                                                      & $o $                           & $\pi $    & $\varpi $                      & $\rho $ & $\varrho $ \\\hline
    \end{tabularx}
    \renewcommand{\arraystretch}{1}

    \caption{使用tabularx搭配模板提供的New Column Type "Y"可以平均分配格子寬度。}
    \label{tab:tabexample4}
\end{table}

\begin{table}[ht]
    \centering
    \renewcommand{\arraystretch}{1.2}

    \begin{adjustbox}{center}
        \begin{tabular*}{1.1\textwidth}{  *6{ c |} @{\extracolsep{\fill}} cccc }
            \multirow{2}{*}{$\varsigma $}     & \multicolumn{9}{c}{這個表格超級超級寬} \\\cline{2-10}
            & $\hbar $       & $\mho $        & $\surd $      & $\bigstar $       & $\ell $        & $\clubsuit $ & $\looparrowleft $       & $\curvearrowright $        & $\upuparrows $ \\
            \hline\hline
            $\gamma $                                       &        $\delta $      & $\epsilon $                    & $\varepsilon $                &      $\zeta $               & $\eta $            & $\theta $            & $\vartheta $                    & $\iota $             & $\kappa $              \\\hline
            $\lambda $                                   &      $\mu $          & $\nu $                 & $\xi $                   &    $o $                 & $\pi $              & $\varpi $              &  $\rho $                   & $\varrho $              & $\sigma $              \\\hline
            $\varsigma $                                  &   $\tau $             & $\upsilon $                     &$\phi $                   &    $\varphi $                & $\chi $              & $\omega $              &   $\Gamma $                & $\varGamma $              & $\Delta $              \\\hline
            $\varDelta $                                         &    $\Theta $           & $\varTheta $                    & $\Lambda $                  &   $\varLambda $                 & $\Xi $              & $\varXi $              &    $\Pi $                & $\varPi $              & $\Sigma $              \\\hline
            $\varSigma $                                 &     $\varUpsilon $                & $\varUpsilon $                     & $\Phi $                   &        $\varPhi $           &$\Psi $             & $\varPsi $              &            $\Omega $      & $\varOmega $            & $\aleph $              \\
            \hline\hline
            $\beth $                                   &        $\daleth $                  & $\gimel $                    & $\varpropto $                   &       $\blacktriangleleft $                & $\curlyeqprec $            & $\precsim $              &      $\Im $               & $\Re $             & $\mho $             \\\hline
        \end{tabular*}
    \end{adjustbox}

    \renewcommand{\arraystretch}{1}

    \caption{這是一個會炸出邊界的超寬表格,使用adjustbox讓表格還是維持置中。}
    \label{tab:tabexample5}
\end{table}

\begin{figure}[hbpt]
    \centering
    \begin{subfigure}{0.45\linewidth}
        \includegraphics[width=\textwidth]{Figures/computer_science.jpg}
        \caption{左邊放個圖片}
    \end{subfigure}
    \hfill
    \begin{subfigure}{0.45\linewidth}
        \includegraphics[width=\textwidth]{Figures/computer_science.jpg}
        \caption{右邊放個圖片}
    \end{subfigure}
    \caption{2張圖擺在一起}
    \label{fig:figexample2}
\end{figure}

\begin{figure}[hbpt]
    \centering
    \begin{subfigure}{0.4\linewidth}
        \includegraphics[width=\textwidth]{Figures/computer_science.jpg}
        \caption{左邊放個圖片}
    \end{subfigure}
    \hfill
    \begin{subfigure}{0.48\linewidth}
        \centering
        \begin{tabular}{c | c }
            $\curvearrowleft $   & $\curvearrowright $  \\
            \hline \hline
            $\circlearrowleft $  & $\circlearrowright $ \\
            $\twoheadleftarrow $ & $\Lleftarrow $       \\
            $\Rrightarrow $      & $\leftarrowtail $    \\
        \end{tabular}
        \caption{右邊放個表格}
    \end{subfigure}
    \caption{圖表擺在一起}
    \label{fig:figexample3}
\end{figure}