%----------------------------------------------------------------------
% 結論與未來展望
%----------------------------------------------------------------------

\chapter{結論與未來展望\small{(如何使用footnote)}}\label{chap:conclusion}

\section{研究成果}

表\ref{tab:tabexample6}。表\ref{tab:tabexample7}。一個註腳\footnote{這是一個正常文字中的footnote}。在人生的歷程中,研究成果的出現是必然的。做好研究成果這件事,可以說已經成為了全民運動。領悟其中的道理也不是那麼的困難。對於一般人來說,研究成果究竟象徵著什麼呢?總而言之,我們普遍認為,若能理解透徹核心原理,對其就有了一定的了解程度。想必大家都能了解研究成果的重要性。我們不得不面對一個非常尷尬的事實,那就是,若沒有研究成果的存在,那麼後果可想而知。不難發現,問題在於該用什麼標準來做決定呢?當你搞懂後就會明白了。研究成果,到底應該如何實現。世界上若沒有研究成果,對於人類的改變可想而知。如果別人做得到,那我也可以做到。

\section{未來展望}

\begin{table}[ht]
    \centering
    \renewcommand{\arraystretch}{1.2}

    \begin{tabular}{ c | c | c | c | C{10em}}
        $\alpha$                               & \multicolumn{2}{c|}{$\gamma $} & \multicolumn{2}{c}{$\delta $}                                   \\\hline
        $\beta$                                & $\epsilon $                    & $\varepsilon $                & $\zeta $     & $\grave{\zeta} $ \\ \hline\hline
        這邊有第一個表格footnote \footnotemark & $\eta $                        & $\theta $                     & $\vartheta $ & $\iota $         \\\hline
    \end{tabular}

    \renewcommand{\arraystretch}{1}

    \caption{第一種表格footnote}
    \label{tab:tabexample6}
\end{table}
\footnotetext{第一種表個footnote,這種方法一個table只能有一個footnote,而且要寫在2個地方。}

\begin{table}[ht]
    \centering
    \renewcommand{\arraystretch}{1.2}

    \begin{tabular}{ c | c | c | c | c}
        \multirow{2}{*}{$\alpha$}                                                                                  & \multicolumn{2}{c|}{$\gamma $} & \multicolumn{2}{c}{$\delta $}                                   \\\cline{2-5}
                                                                                                                   & $\epsilon $                    & $\varepsilon $                & $\zeta $     & 這是一個內容有點 \\ \hline\hline
        這邊有第二個表格footnote \tablefootnote{這邊有另外一種table footnote,}                                    & $\eta $                        & $\theta $                     & $\vartheta $ & $\iota $         \\\hline
        這邊有第三個表格footnote \tablefootnote{使用\textbackslash tablefootnote就可以在table內產生多組footnote,} & $\kappa  $                     & $\lambda  $                   & $\mu  $      & $\nu  $          \\\hline
        這邊有第四個表格footnote \tablefootnote{但是也會讓你的table部分的code變得有點亂。}                         & $\xi  $                        & $o  $                         & $\pi  $      & $\varpi  $       \\\hline
    \end{tabular}

    \renewcommand{\arraystretch}{1}

    \caption{第二種表格footnote}
    \label{tab:tabexample7}
\end{table}


% \footnotetext{使用\textbackslash footnotemark和\textbackslash footnotetext對表格文字做footnote}